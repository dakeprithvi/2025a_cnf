\documentclass[fontsize=11pt]{article}
\usepackage[margin=0.75in]{geometry}
\usepackage{color}
\usepackage{graphicx}
\usepackage{float}
\usepackage{polynom}
\usepackage{placeins}
\usepackage{longtable}
\usepackage{gensymb}
\usepackage{tikz}
\tikzset{every picture/.style={font=\normalsize}}
\DeclareOldFontCommand{\rm}{\normalfont\rmfamily}{\mathrm}
\DeclareOldFontCommand{\it}{\normalfont\rmfamily}{\mathrm}
\usepackage{mathtools}
\usepackage{hyperref}
\hypersetup{colorlinks=false,
            pdfborder={0 0 0}}
\renewcommand{\thefootnote}{\alph{footnote}}
\newcommand{\R}{\mathbb{R}}
\usepackage{amsmath, bm}   % Extra math commands and environments from the AMS
\usepackage{amssymb}   % Special symbols from the AMS
\usepackage{amsthm}    % Enhanced theorem and proof environments from the AMS
\usepackage{latexsym}  % A few extra LaTeX symbols
\usepackage{rxn}
\usepackage[version=4]{mhchem}
\usepackage{algorithm, algpseudocode}
\usepackage{cleveref}
\usepackage{url}
\newcommand{\norm}[1]{\lVert #1 \rVert}
\providecommand{\email}{}
\renewcommand{\email}[1]{\texttt{#1}}
\usepackage{caption}
\captionsetup{labelfont=bf, labelsep=period, labelformat=simple, labelsep=colon, figurename=Fig.}
\usepackage[authoryear, round]{natbib}
\newtheorem{theorem}{Theorem}
\newtheorem{lemma}[theorem]{Lemma}
\newtheorem{proposition}[theorem]{Proposition}
\newtheorem{conjecture}[theorem]{Conjecture}
\newtheorem{corollary}[theorem]{Corollary}

\theoremstyle{definition}
\newtheorem{definition}[theorem]{Definition}
\newtheorem{remark}[theorem]{Remark}
\newtheorem{assumption}[theorem]{Assumption}
\newtheorem{hypothesis}[theorem]{Hypothesis}
\newtheorem{property}[theorem]{Property}

%\newcommand{\tL}{\tilde{L}}

\graphicspath{{./}{./figures/}{./figures/paper/}}

\setcounter{topnumber}{2}              %% 2
\setcounter{bottomnumber}{1}           %% 1
\setcounter{totalnumber}{3}            %% 3
\renewcommand{\topfraction}{0.9}       %% 0.7
\renewcommand{\bottomfraction}{0.9}    %% 0.3
\renewcommand{\textfraction}{0.1}      %% 0.2
\renewcommand{\floatpagefraction}{.7}  %% 0.5

\title{\bf{Neural Ordinary Differential Equations}}
\author{Prithvi Dake\\
Department of Chemical Engineering, University of California, Santa Barbara}
\date{\textit{ME255NN: Neural Networks for modeling, optimization and control} \\
\textit{Winter 2025}}

\begin{document}

\maketitle
\section{Introduction}

Conventional neural network architectures, such as recurrent neural networks 
(RNNs) and residual networks (ResNets), can be viewed as discrete approximations
of ordinary differential equations (ODEs), specifically resembling the explicit
Euler scheme. Neural ODEs (NODEs) extend these architectures into a continuous 
framework, providing a more principled way of modeling dynamical systems.

One major challenge in training NODEs is gradient propagation during backpropagation.
Unlike traditional architectures, where intermediate states are stored for
gradient computation, NODEs require solving an augmented ODE system, including 
sensitivity equations, to compute gradients efficiently. This adjoint sensitivity 
method significantly reduces memory requirements compared to brute-force 
backpropagation. Overcoming this training limitation enables NODEs to be 
applied in broader contexts, such as continuous normalizing flows and state-space 
models.

In this review, we cover fundamental concepts and recent advancements in NODEs. 
We begin with a discussion on artificial neural networks (ANNs) implemented in 
a physics-informed manner that does not involve recurrence or differential 
modeling as shown by \cite{raissi:perdikaris:karniadakis:2019}. Next, we explore black-box 
and discrete data-driven models for process systems using ResNets and RNNs. 
Building on these foundations, we introduce sensitivity and adjoint equation 
methods referring \cite{stapor:froehlich:hasenauer:2018}, and then introduce the 
formal development of NODEs by \cite{chen:rubanova:bettencourt:duvenaud:2018}. 

We then review advanced applications, including generative latent models by
\cite{rubanova:chen:duvenaud:2019}, continuous normalizing flows by
\cite{grathwohl:chen:bettencourt:sutskever:duvenaud:2018}, and robustness 
analyses of NODEs by \cite{yan:du:tan:feng:2019}. Finally, we discuss efficient 
NODE training techniques compiled in \cite{finlay:jacobsen:nurbekyan:oberman:2020}.

The structure of this report is as follows: We first introduce conventional 
neural network architectures, such as RNNs and ResNets, and their connection 
to NODEs. Next, we explain how backpropagation is handled in NODEs and discuss 
applications in continuous normalizing flows and NODE-based state-space models. 
This is followed by case studies demonstrating practical implementations. 
The report concludes with a summary of key findings and an overview of ongoing 
developments in the field.

\section{Background}

\subsection{ANNs for Physics-Informed Learning}

In most engineering systems, we only have access to noisy measurements of 
input-output data, typically originating from high-dimensional, nonlinear dynamic 
systems. While such systems can be approximated using finite impulse response (FIR) 
or autoregressive exogenous (ARX) models, these approaches often fail to respect 
fundamental physical laws. One way to address this limitation is by penalizing 
violations of these laws—a key idea behind physics-informed neural networks (PINNs), 
as introduced by \cite{raissi:perdikaris:karniadakis:2019}.

However, it is important to view PINNs not merely as data-driven models but 
rather as a solution strategy for PDEs and ODEs. Fundamentally, they operate 
as \textit{methods of weighted residuals}—specifically, collocation schemes—where 
the residual is minimized at selected collocation points. While PINNs can be 
used to learn certain model parameters, their results can be difficult to interpret 
without a fundamental understanding of the underlying physics.

Recent research has leveraged the PINN framework to enforce simple mass and energy 
balance equations at steady-state, as this is often the only available information 
in practical scenarios. \Cref{alg:PINNs} outlines the basic algorithm for training 
a PINN applied to an ODE system.

\begin{algorithm}[h]
\caption{Training a Physics-Informed Neural Networks (PINNs)}
\label{alg:PINNs}
\begin{algorithmic}[1]
\State \textbf{Input:} Neural network architecture $f_{\theta}$
\State \textbf{Physical Model:} $Lu=f \quad B_1 u = 0 \quad B_2 u = 0$
\State \textbf{Initialize} neural network parameters $\theta$
\Repeat
    \State $\hat{u} = f_{\theta}(x,t)$
    \State $V_{ODE} = \norm{L\hat{u} - f}_{2}$ 
    \Comment{Prediction error minimization} 
    \State $V_{BC} = \norm{B_1 \hat{u}}_{2} + \norm{B_2 \hat{u}}_{2}$
    \Comment{Enforce boundary conditions}
    \State $V = \lambda_1 V_{PDE}  + \lambda_2 V_{BC}$
    \Comment{Combine loss terms}
    \State $\theta \leftarrow \theta - \eta \nabla_{\theta} \mathcal{L}$
    \Comment{Update $\theta$ using gradient descent}
\Until{convergence criteria is met}
\State \textbf{Return} trained model $f_{\theta}$
\end{algorithmic}
\end{algorithm}
    
\subsection{ResNets and RNNs for Process Systems}

Most engineering systems operating near a steady state can be effectively modeled 
using discrete linear time-invariant (DLTI) systems. These state-space models are 
particularly well-suited for systems experiencing small perturbations.
For highly nonlinear systems, ResNets and RNNs offer a similar state-space modeling 
approach. In the system identification community, ResNet and RNN models are commonly 
employed for this purpose, as illustrated in \cref{eq:resrnn}.

\begin{align}
    \hat{x}^+ &= \hat{x} + f_{\theta_{NN}} (\hat{x}, u) &\qquad \hat{x}^+ = f_{\theta_{NN}} (\hat{x}, u)\notag \\ 
    \hat{y} &= g_{\theta_{NN}} (\hat{x}, u) &\qquad \hat{y} = g_{\theta_{NN}} (\hat{x}, u)
    \label{eq:resrnn}
\end{align}

There are two approaches to learning the parameters $\theta_{NN}$ in \cref{eq:resrnn}: 
\textit{multi-step prediction error minimization} and \textit{single-step prediction 
error minimization}. 
The single-step approach minimizes the prediction error at each time step independently, 
while the multi-step approach considers the accumulated error over multiple time steps. 
In practice, the multi-step method tends to be more stable, especially in the presence 
of noisy measurements.

Consider following \textit{true} model of linear chemical kinetics taking place
in a well-mixed batch reactor with $c = (c_A, c_B, c_C)^T$ and initial concentration
$c_0$:
\begin{rxn*}{} 
A \rlh[k_1][k_{-1}] 2 B, \qquad  B \rarrow[k_2] C
\label{rxn:atobtoc}
\end{rxn*}
\begin{gather}
\begin{bmatrix} \dot{c}_A \\ \dot{c}_B \\ \dot{c}_C \end{bmatrix}
=\begin{bmatrix}
    -k_{1} & k_{-1} & 0 \\
    2k_{1} & 2k_{-1} - k_2 & 0 \\
    0 & k_{2} & 0
\end{bmatrix}
\begin{bmatrix}  c_A \\ c_B \\ c_C \end{bmatrix} 
\label{eq:atobtoc}
\end{gather}
Assume we have full-state measurement. Thus, a ResNet model can be easily trained
to predict the concentration profile over time as shown in \cref{fig:resnet}. A similar
example can be reproduced using an RNN model. Note we used multistep prediction error
minimization. One disadvantage of such iterative models is that they require large memory
to store intermediate states for gradient computation during backpropagation (since we basically
apply chain-rule).

\begin{figure}
    \centering
    \includegraphics[width=0.55\textwidth, page=2]{ABC_plot.pdf} 
    \caption{Concentration profiles of species A, B, and C in a batch reactor. The 
    discrete ResNet model does quite a good job to predict the concentration profiles.}
    \label{fig:resnet}
\end{figure}

\subsection{Forward and backward Sensitivity/Adjoint methods}

\subsubsection{Forward Sensitivity}
\label{sec:sens}

Here we give a quick overview of both the forward/backward sensitivity equations for ODEs. Consider following
non-linear state-space model:

\begin{align}
    \dot{x} &= f(x, u; \theta) \qquad x(0) = g(x_0; \theta)\notag \\
    y &= g(x, u; \theta)
    \label{eq:ssode}
\end{align}

Assume $y=x$ i.e. full state measurement and hence we also know the initial condition
$x_0$. Consider a least-squares objective to measure the fit to the data:

\begin{align}
    \min\limits_{\theta} V(x) = \norm{x - \hat{x}}^2 
\end{align}
To propagate the gradient we need to solve the sensitivity equations:
\begin{align}
    \dfrac{\partial V(x)}{\partial \theta_j} &= -2 \sum_i (\hat{x} - x) \dfrac{\partial x_i}{\partial \theta_j} \notag \\ 
    S_{ij} &= \dfrac{\partial x_i}{\partial \theta_j}
    \label{eq:sens}
\end{align}
where $S_{ij}$ is the sensitivity of model solution with parameter. We differentiate $S_{ij}$ to obtain:
\begin{align}
    \dfrac{d S_{ij}}{dt} &= \dfrac{d}{dt} \left(\dfrac{\partial x_i}{\partial \theta_j}\right) =
    \dfrac{\partial}{\partial\theta_j} \left(\dfrac{d x_i}{dt}\right) = 
    \dfrac{\partial }{\partial \theta_j} f_i(x; \theta) 
\end{align}
Using chain rule on $f$, we get following matrix equation:
\begin{align}
    \dfrac{dS}{dt} = f_xS + f_{\theta} \qquad  S(0) = g_\theta
\end{align}
We could easily calculate the initial condition 
if we are using a state estimator by using the \cref{eq:sens}. Thus, we can construct
following augmented ODE system:
\begin{align}
    \begin{bmatrix} \dot{x} \\ \dot{S} \end{bmatrix} = 
    \begin{bmatrix} f(x, u; \theta) \\ f_xS + f_{\theta} \end{bmatrix}
    \qquad
    \begin{bmatrix} x(0) \\ S(0) \end{bmatrix} =
    \begin{bmatrix} x_0 \\ g_\theta \end{bmatrix}
    \label{eq:augsens}
\end{align}
This augmented system can be solved using any ODE solver, allowing gradient 
propagation. However, if $\theta \in \R^{p}$ and $x \in \R^{n}$, we need to 
solve $p \times n + 1$ equations i.e. the cost increases with $p$. Thus, forward
sensitivity is not an efficient method to propagate gradients. However, we can circumvent
the problem of storing intermediate states using backward sensitivity method discussed next. This key advantage 
of backward adjoint method contributed to the popularity of Neural ODEs (NODEs), beyond their interpretation 
as a continuous counterpart to Euler-step-based architectures like ResNets. We also show 
a short example on solving the augmented ODE system for forward sensitivity. Consider, 
following kinetics:
\begin{align}
    \dot{c}_A &= -kc_A \notag \\
    c_A(0) &= c_{A0}
\end{align}
Say we want to calculate sensitivities wrt $k, c_{A0}$. With reference to \cref{eq:augsens},
for $S_1 = \dfrac{\partial c_A}{\partial k}$ and $S_2 = \dfrac{\partial c_A}{\partial c_{A0}}$,
we have following augmented system:
\begin{align}
    \begin{bmatrix} 
        \dot{c}_A \\
        \dot{S_1} \\ 
        \dot{S_2} 
    \end{bmatrix} = 
    \begin{bmatrix}
        -kc_A \\
        -kS_1 - c_A \\
        -kS_2 \end{bmatrix}
    \qquad
    \begin{bmatrix} 
        c_A(0) \\ 
        S_1(0) \\ 
        S_2(0) 
    \end{bmatrix} =
    \begin{bmatrix} 
        c_{A0} \\ 
        0 \\ 
        1
    \end{bmatrix}
\end{align}
If we were learning the parameter $k$ and initial condition $c_{A0}$, we could
show a gradient descent update as follows:
\begin{align}
    \begin{bmatrix} k \\ c_{A0} \end{bmatrix} \leftarrow
    \begin{bmatrix} k \\ c_{A0} \end{bmatrix}
    - \eta 
    \begin{bmatrix} 
        -2 (\hat{c}_A  - c_A)^T S_1 \\ 
        -2 (\hat{c}_A  - c_A)^T S_2
    \end{bmatrix}
\end{align}

\subsubsection{Backward Sensitivity}

In forward sensitivity since we solve the augmented ODE system, we do not need 
to worry about time-steps taken by the ODE solver. However for backward sensitivity,
it becomes a major problem. Here we show derivation of the adjoint equations for the
continuous version rather than the previous discrete version (where the loss is obtained by summing).
Consider following least-squares objective for \cref{eq:ssode} and the Lagrange functional:

\begin{align}
    V(x) &= \int_0^T (y - \hat{y})^2 dt = \int_0^T g(x, u; \theta)dt \notag \\
    \mathcal{L} &= \int_0^T g dt + \int_0^T \lambda(t)^T \left(f - \dfrac{dx}{dt}\right) dt
\end{align}
Hence,
\begin{align}
    \dfrac{\partial \mathcal{L}}{\partial \theta} &= \int_0^T \dfrac{\partial g}{\partial x^T} \dfrac{\partial x}{\partial \theta^T} dt + 
    \int_0^T \lambda^T \left(\dfrac{\partial f}{\partial x} \dfrac{\partial x}{\partial \theta^T} + \dfrac{\partial f}{\partial \theta} - \dfrac{\partial}{\partial \theta}\dfrac{dx}{dt}\right) dt \notag \\
    \dfrac{\partial \mathcal{L}}{\partial \theta} &= \int_0^T \dfrac{\partial g}{\partial x^T} \dfrac{\partial x}{\partial \theta^T} dt + 
    \int_0^T \lambda^T \left(\dfrac{\partial f}{\partial x} \dfrac{\partial x}{\partial \theta^T} + \dfrac{\partial f}{\partial \theta} - \dfrac{d}{dt}\dfrac{\partial x}{\partial \theta^T}\right) dt \notag \\
    \dfrac{\partial \mathcal{L}}{\partial \theta} &= \int_0^T \lambda^T \dfrac{\partial f}{\partial \theta} dt + \int_0^T \left(\dfrac{\partial g}{\partial x^T}
    + \lambda^T \dfrac{\partial f}{\partial x} - \lambda^T\dfrac{d}{dt}\right)\dfrac{\partial x}{\partial \theta^T} dt
\end{align}
The term $\int_0^T{\lambda^T \dfrac{d}{dt} \dfrac{\partial x}{\partial \theta^T}dt}$ 
can be solved using integration by parts. Thus, we have:

\begin{align}
    \dfrac{\partial \mathcal{L}}{\partial \theta} &= \int_0^T \lambda^T \dfrac{\partial f}{\partial \theta} dt + \int_0^T \left(\dfrac{\partial g}{\partial x^T}
    + \lambda^T \dfrac{\partial f}{\partial x} - \dfrac{d\lambda}{dt}\right)\dfrac{\partial x}{\partial \theta^T} dt - \lambda^T(0)\dfrac{\partial x}{\partial \theta^T}|_0
    + \lambda^T(T)\dfrac{\partial x}{\partial \theta^T}|_T 
    \label{eq:adjoint}
\end{align}

Now, the whole objective of backward sensitivity is to avoid calculating $\dfrac{\partial x}{\partial \theta^T}$.
Also note that the equality constraint in the Langrange functional is always satisfied
since we solve the ODE system. Thus we get to choose $\lambda$ such that the terms
associated with $\dfrac{\partial x}{\partial \theta^T}$ vanish. Also, we can 
easily calculate $\dfrac{\partial x}{\partial \theta^T}|_0$ but we would like to set
$\dfrac{\partial x}{\partial \theta^T}|_T = 0$. Thus, we can construct following adjoint 
equations:

\begin{align}
    \dfrac{d\lambda}{dt} &= \left(\dfrac{\partial g}{\partial x^T} + \lambda^T \dfrac{\partial f}{\partial x}\right)^T \notag \\
    \lambda(T) &= 0
\end{align}
Referring to \cref{eq:adjoint}, we see $\dfrac{\partial V}{\partial \theta} = \dfrac{\partial \mathcal{L}}{\partial \theta} + \lambda^T(0)\dfrac{\partial x}{\partial \theta^T}|_0$.
Thus, we do no need to backprop but just solve the augmented ODE system once forward 
and once backward in each optimal iteration unlike ResNets or RNNs. Also, note that the number 
of adjoint equations do not increase with parameters. However, such a method is not 
an ultimate panacea. Such systems are still vulnerable to the problems commonly faced by
numerical ODE solvers like stiffness, accuracy, etc. Sometimes we may end up 
with noisy gradients which may lead to
poor convergence.

\subsection{Neural Ordinary Differential Equations}

With the background established in \cref{sec:sens}, we can now introduce the algorithm
for training NODEs. The key idea is to replace the right-hand side of the ODE system
with a neural network. With auto-differentiation we can easily construct the augmented
ODE system and solve it using any ODE solver. 

\begin{algorithm}[h]
\caption{Training a Neural Ordinary Differential Equation (NODE)}
\label{alg:NODE}
\begin{algorithmic}[1]
\State \textbf{Input:} Neural network architecture $f_{\theta}$
\State \textbf{ODE:} 
$
    \begin{bmatrix} \dot{x} \\ \dot{S} \end{bmatrix} = 
    \begin{bmatrix} f_\theta(x, u; \theta) \\ f_xS + f_{\theta} \end{bmatrix}
    \qquad
    \begin{bmatrix} x(0) \\ S(0) \end{bmatrix} =
    \begin{bmatrix} x_0 \\ g_\theta \end{bmatrix}
    \label{eq:augsens}
$
\Comment{Construct the augmented ODE system say, using \texttt{PyTorch}}
\State \textbf{Initialize} neural network parameters $\theta$
\Repeat
    \State $\hat{x}_{aug} = \text{ODESolve}(f_{aug}, x_{aug}(0), t)$
    \State $V = \norm{\hat{x} - x}_{2}$
    \State $\theta \leftarrow \theta - \eta \nabla_{\theta} \mathcal{L}$
\Until{convergence criteria is met}
\State \textbf{Return} trained model $f_{\theta}$
\end{algorithmic}
\end{algorithm}
We close the section with a same example as shown in \cref{eq:atobtoc} but now
we use a NODE to predict the concentration profiles. The results are shown in
\cref{fig:node}.

\begin{figure}[h]
    \centering
    \includegraphics[width=0.55\textwidth, page=1]{ABC_plot.pdf} 
    \caption{Concentration profiles of species A, B, and C in a batch reactor. The
    NODE model also does quite a good job to predict the concentration profiles. 
    In fact for such simple problem, we hardly see any difference between ResNet and NODE.}
    \label{fig:node}
\end{figure}

\subsection{Generative latent models based on continuous normalizing flows}

\begin{figure}[h]
    \centering
    \includegraphics[width=1\textwidth, page=3]{ABC_plot.pdf} 
\end{figure}

\subsection{Robustness analysis of NODEs}

\subsection{Efficient training techniques for NODEs}

\section{Conclusions}


\newpage
\FloatBarrier
\bibliographystyle{abbrvnat}
\bibliography{cnf_bib, abbreviations,articles,books,unpub,proceedings}
\end{document}
